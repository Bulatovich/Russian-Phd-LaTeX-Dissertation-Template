\chapter{Оформление различных элементов}\label{ch:ch1}

\section{Форматирование текста}\label{sec:ch1/sec1}


\section{Ссылки}\label{sec:ch1/sec2}

%вызывает biblatex warning относительно опции sortcites, потому что неясно, к
%какому источнику относится уточнение о страницах, а bibtex об этой проблеме
%даже не предупреждает
% Следует обратить внимание, что пробел после запятой внутри \cite{}
% обрабатывается ожидаемо, а пробел перед запятой, может вызывать проблемы при
% обработке ссылок.

% Тут специально написано по-разному тире, для демонстрации, что
% применение специальных тире в настоящий момент в biblatex приводит к непоказу
% "с.".










\section{Формулы}\label{sec:ch1/sec3}


\subsection{Ненумерованные одиночные формулы}\label{subsec:ch1/sec3/sub1}





%Все \original... команды заранее, ради этого примера, определены в Dissertation\userstyles.tex


\subsection{Ненумерованные многострочные формулы}\label{subsec:ch1/sec3/sub2}










\subsection{Нумерованные формулы}\label{subsec:ch1/sec3/sub3}






\subsection{Форматирование чисел и размерностей величин}\label{sec:units}












\subsection{Заголовки с формулами: \texorpdfstring{\(a^2 + b^2 = c^2\)}{%


\section{Рецензирование текста}\label{sec:markup}











\section{Работа со списком сокращений и~условных обозначений}\label{sec:acronyms}




% запись сокращения в список происходит командой \nomenclature,
% а не употреблением самого сокращения

